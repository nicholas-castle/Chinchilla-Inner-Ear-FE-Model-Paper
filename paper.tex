\documentclass[12pt]{article}
\usepackage[margin=1.0in]{geometry}
\usepackage{graphicx}
\usepackage{multicol}
\usepackage{float}
\graphicspath{{./images/}}
\usepackage{etoolbox}
\usepackage{longtable}
\usepackage{titlesec}
\setlength{\parskip}{0pt}


\usepackage{setspace}
\doublespacing

\usepackage{xr}
\externaldocument{figures}

\titleformat*{\section}{\normalsize\bfseries\raggedright}
\titleformat*{\subsection}{\normalsize\bfseries\raggedright}
\titleformat*{\subsubsection}{\normalsize\bfseries\raggedright}
\makeatletter
\renewcommand*{\@seccntformat}[1]{\csname the#1\endcsname\hspace{0.25cm}}
\makeatother

\makeatletter
\providecommand{\institute}[1]{% add institute to \maketitle
\apptocmd{\@author}{\end{tabular}
\par
\smallskip
\begin{tabular}[t]{c}
#1}{}{}
}
\makeatother


\title{\textbf{\normalsize 3D Multisystemic Finite Element Model of the Chinchilla Inner Ear}}
\author{\normalsize Nicholas Castle, Junfeng Liang, Ke Zhang, Chenkai Dai}
\institute{\normalsize Aerospace and Mechanical Engineering, University of Oklahoma, Norman, OK 73019 \\ \normalsize United States of America}


\date{}
% Hint: \title{what ever}, \author{who care} and \date{when ever} could stand 
% before or after the \begin{document} command 
% BUT the \maketitle command MUST come AFTER the \begin{document} command! 
\begin{document}

\maketitle

\noindent Corresponding author: \\
Chenkai Dai, Ph.D. \\
Associate Professor \newline
Department of Aerospace and Mechanical Engineering \newline
University of Oklahoma \newline
865 Asp Ave, \newline
Norman, OK 73019 \newline
Phone: (405)325-3234 \newline
E-mail: Chenkaidai@ou.edu \newline

\newpage

\begin{abstract}
The mammalian inner ear consists of the cochlea and vestibular systems. These systems act as biomechanical sensors responsible for converting acoustic waves, linear translation, and angular motion experienced by the head into sensible impulses encoded in 3D space. This paper presents current progress towards a 3D FE model of the chinchilla inner ear capable of both mechanical and electrical analysis. This includes a functional mechanical model of the inner ear with response evaluable at the basilar membrane, cupulas, and vestibular maculae. Geometric results of modeling the cochlear, vestibular, and facial nerves are also presented. Accurate modeling of this region is accomplished through the use of $\mu$MRI and $\mu$CT scanning technology with details verified by literature research. This follows a protocol for rapid modeling which is presented here in its current form. Results from harmonic acoustic simulation and simulation of both linear and angular acceleration are shown and compared with those received from clinical studies to verify the models integrity. \newline \newline \textbf{Keywords:} inner ear, vestibular, cochlea, chinchilla, labyrinth, finite element, mechanics \newpage
\end{abstract}


\section{Introduction}
The general morphology of the inner ear is constant among most mammals and is the result of millions of years of evolution. \cite{ekdale:chinchanat} The cochlea and vestibular system, its primary sensory organs, provide the senses of hearing and balance by detecting the movement of endolymphatic fluid through an interconnected series of lumens called the membranous labyrinth using sensory hair cells. An accurate representation of the structures which support or otherwise effect the mechanical and electrical function of these organs is key to developing a finite element model of the mammalian inner ear which can accurately simulate sound transmission and balance function for both typical ears and ears with various pathologies. 

The codependant relationship of the vestibular system and cochlea is well documented. For example, sensorineural hearing loss commonly presents alongside impairments in balance and loss of vestibular function often sees delays in major gross motor milestones in young children. \cite{cushing} A number of FE models have been constructed of the inner ear, typically focusing on either the cochlea, vestibular system, or vestibulocochlear nerve. De Paolis et al.’ study (2017) detailed a 3D human cochlea model based on microCT imaging made for studying perilymph hydrodynamics. Elliott et al. Zhang and Gans made a model complete with coiled cochlea, middle ear, and ear canal (Zhang et al., 2011). The majority of fe models made of the vestibular system focus singularly on the pressures around or sensorinueral reactions of the cupulas,(Yu et al., 2018; Zdravkovic et al., 2017, Hedjoudje et al., 2019; Schier et al., 2018) Many of these studies simplify the complex geometry of the cochlea and vestibular system into geometric shapes for mechanical analysis. This can be valuable as it allows for easy reproduction of results however it does not allow for accurate transient or electrical analysis, both relying heavily upon the topology of the model for calculations. In a clinical setting this difference can be seen in the unforeseen effects of cochlear implants, including facial numbness and hearing/balance loss, due in part to the proximity of the vestibulocochlear nerve to the facial nerve. (Miwa et al., 2019)


This study focuses on obtaining a 3D finite element model of the chinchilla membranous labyrinth, bony labyrinth, cochlea, brain, temporal bone and related vestibular, cochlear, and facial nerve pathways. By coreferencing $\mu$MRI and $\mu$CT scans to detect soft and osseous tissues it becomes possible to create a model capable of conducting both mechanical and electrical analysis, the first step toward fe modeling which can analyze both sensorinueral and mechanical causes of hearing loss. The mechanical model will be completed first and compared to expected response curves (cite) to demonstrate its initial validity.
   
%   The innermost lumens of the cochlea and vestibular system are part of a network of lymphatic fluid filled chambers called the membranous labyrinth. This includes the scala media of the cochlea, and the utricular vestibule, saccular vestibule, and three semcicircular canals  (SCC’s) of the vestibular system. The membranous labyrinth contains endolymphatic fluid whose movement stimulates the sensory hair cells in both systems generating the sensible electronic impulses sent to the brain which represent hearing and balance. These lumens are invariably surrounded by a layer of perilymphatic fluid whose outer boundary is the bony labyrinth, primarily consisting of osseous tissue as the name implies. In both systems this fluid plays a vital role in dampening the noise in vibrations received from the inner ear and angular motion of the head. This provides a sort of dampening effect and ensures that normal movement results in minimal effect on the hearing and balance of vertebrates. This is especially important in the case of the cochlea as even small ambient changes in osmolality and pressure can result in permanent hearing loss, tinnitus, and vertigo.
%-keep
%
%
%Endolymphatic fluid is produced in the endolymphatic sac near the surface of the temporal bone and travels through the endolymphatic duct to the branching connection of the utricular vestibule to the saccular vestibule. Disruptions of this specific system are suspected to cause a variety of hearing disorders due to variations in pressure and direction of fluid flow.
%Impulses in both systems are stimulated by the movement of endolymphatic fluid due to displacement of the stapes footplate in the case of the cochlea and displacement from acceleration in the case of the vestibular system. Impulses generated by the cochlea and vestibular system are sent to the brain via the many branches of the cochlear and vestibular nerves. These nerves are together referred to as the vestibulocochlear nerve, a nerve bundle running tangent to the facial nerve and attaching to the brain at the brain stem.    
%-shorten


%The cochlea is a coil-like organ which, according to the efficient packing hypothesis, allows it to occupy minimal space while providing maximum range and accuracy in translation of vibrations to sensible impulses sent to the brain. \cite{ekdale:chinchanat} The cochlea’s functional apparatus is concentrated in its inner lumen which is divided into three levels. The central level, or scala media, contains the basilar membrane where auditory hair cells are anchored and whose movement provide the sense of hearing. Movement of the endolymphatic fluid in the cochlea is primarily caused by the piston-like movement of the stapes footplate in the middles ear, driven by the reactions of the auditory ossicles from vibrations in the tympanic membrane, themselves caused by sound waves in the air. \cite{wang:femodel} The movement of these hair cells creates electronic impulses relayed to the cochlear nerve, attached to the inner rim of the basilar membrane, and move through the cochlear nerve at the center of its helical structure to the brain. \cite{ni:cochmcx}
%-keep


%The vestibular system is composed of the three semicircular canals (SCC’s), the utricular vestibule, and the saccular vestibule, and provides the sense of balance. The semicircular canals, toroidal in shape, are each oriented nearly orthogonal to each other providing a full picture of the angular acceleration of the head. Movement of the head results in displacement of the lymphatic fluid contained within the semicircular canals whose movement is captured by sensory hair bundles effected by its flow. These sensory hair bundles are concentrated in the cupula at the base of each SCC, themselves contained in enlarged regions of the membranous labyrinth called ampullas. 
%-keep

%The cupulas, spanning the breadths and heights of the ampullas, are anchored at their bases to the outer membrane of the membranous labyrinth where the branches of the vestibular nerve attach. This region is known as the crista and is visible as a small ridge in the model. Hair cells rise from this point up into the cupula and are excited by stimuli according to their angle of deflection due to mechotransduction of endolymphatic movement. These hair cells are much longer than those found in the cochlea to increase sensitivity to small movements. This results in sensible electronic impulses sent through the attached branches of the vestibular nerve to the brain stem. \cite{rabbitt}
%-shorten?


%Oftentimes the cochlea and vestibular system are modeled independently however recent studies have shown that these two systems are codependent. 
%-Keep, move some to limitations section
%
%
%-keep, use a reference for anatomy
%
%
%-keep
%
%
%
%
%
%
%
%-Junfeng- \newline
%
%-Junfeng-
%ADD
%- Main limitations
%- Current models don't use both the cochlea and vestibular system
%- Can I reference Junfengs paper (not necessarily on this topic), will it be published soon?
%- Focus on geometric limitations because we are only using basic simulations
%-




\section{Methods}
\subsection{Data Source}
Model geometry was generated through 3D reconstruction of ultra-high resolution $\mu$CT and $\mu$MRI data sets. CT scans were acquired at 12 $\mu$m voxel size and MRIs acquired at 30 and 48 $\mu$m voxel size. Achieving this voxel size with adequate reduction of feedback for the MRI required 26 hours of acquisition in an 11.7 Tesla magnet. MRI and CT data was needed to obtain adequate resolution in both bone tissues and soft tissues.

Figure \ref{segment2} shows a representative MRI image in the frontal plane where all major features are labeled and visible, illustrating the complexity of the mammalian inner ear due to its many intertwining nerves and small separation between wave conducting volumes. Notable inclusions are the facial nerve, cochlear nerve, and vestibular nerve.

Figure \ref{ctsegment} shows a CT image segmented in the frontal plane with notable features annotated. These include the vestibular lumen, superior SCC and its ampulla, and the assumed location of the endomlyphatic sac due to a divot in the temporal bone in the opproximate location seen in humans.\cite{ensac}


\subsection{Segmentation}
The structure of the chinchilla bulla was segmented using a program called 3D Slicer. \cite{slicer} This was accomplished principally by using a seeding algorithm to expand small selections based on differences in surrounding tissue densities. This process was repeated for each image in the received $\mu$MRI and $\mu$CT data. This resulted in the distinct regions denoting varying tissues seen in Figures \ref{segment2} and \ref{ctsegment}.

Nervous tissue proved the most difficult to accurately select, however clear divisions between osseous, epithelial, and nervous tissues could still be seen in many images. The appearance of tissues was brightened significantly in the intervening images allowing for accurate outlines of nerves to be reconstructed.

The intervening space between each captured imaged was interpolated using an algorithm native to the 3D Slicer program yielding a full 3D point cloud capable of being turned into a modifiable .stl file. \cite{slicer} This could be viewed and modified in 3D space during the segmentation process allowing easy removal of artifacts and quick visual verification of operations.


\subsection{Geometry}
The point clouds obtained from $\mu$MRI imaging were imported as .stl files into MeshMixer for smoothing of the outer surfaces as well as construction of the membranous labyrinth based on extrapolation of the imported data. Artifacts which resulted in additional unattached structures were removed leaving only the desired anatomy in place. Holes and gaps in the semicircular canals and cochlea were repaired manually to ensure proper connectivity between surfaces. Curvatures of repaired sections were matched to that of surrounding surfaces using tools native to the MeshMixer program.

The membranous labyrinth was refined in MeshMixer by creating a copy of the bony labyrinth, shrunk by a fraction to create two volumes, one enclosed within the other. The precise structure of the utricle and saccule are not concretely known due to their small size and thin membranous structure, creating high potential for damage in histological studies and difficulty in imaging for even $\mu$MRI technologies. The utricle was modeled as a slightly shrunken version of the portion of the surrounding bony labyrinth anterior to the cupulas. Cupulas structures are based off of the work conducted by Dr. Rabbitt in “Semicircular canal biomechanics in health and disease” published in 2019. \cite{rabbitt} They are modeled as spanning the majority of the cupula of each SCC. The cross-sectional area and position of the membranous labyrinth within the SCC’s was corrected using the figures shown in Iversen, M. M., Rabbitt, R. D. (2017). \cite{iversen} This was accomplished by sketching cross sections in SolidWorks, placing them appropriately within the model of the bony labyrinth smoothed in MeshMixer, and creating a thin layer element which dynamically connects each cross section. The utricle and saccule were modeled to mirror that seen in Fig. 3 of W. W., et al.’s study titled “The Endolymphatic Duct and Sac” published in 1997.

The cochlea was modeled primarily based on information obtained directly from MRI imaging using the previously described segmentation process. Characteristic ridges on the surface of the bony labyrinth allowed curves to be drawn along the outside of the cochleas spiraling structure. These curves were connected by planes forming a wedge whose superior face represents Reissner’s Membrane, and whose inferior face represents the basilar membrane. This shape was compared with that seen in literature and was confirmed to the have correct structure. \cite{trevino:chinchmodel} \cite{ekdale:chinchanat} \cite{ni:cochmcx} The cochlea was scaled until the basilar membrane was approximately the average length in chinchillas of 18.3 mm along its midline. (Bohne and Carr, 1979; Hardy, 1938) 
 
The rough geometries of the vestibular, cochlear, facial nerves, and auditory cortex were inserted into the initial model of the bony labyrinth and smoothed to have consistent surfaces. The attachment point between brain and nerve was left largely unaltered from original data. The model of the temporal bone, received from $\mu$CT imaging, was inserted into the model to further inform nerve and cochlear geometry. The cochlear nerve was refined by enlarging it to completely fill the space within the spiral-shape of the cochlear membranous labyrinth, using Boolean subtraction of the membranous labyrinth to create a mostly uniform boundary. These efforts were compared with the work of Romano, N., Federici, M., Castaldi, A. in 2019 to ensure accuracy in nerve connectivity and path. \cite{romano:nerves}
 
\subsection{Meshing}
Only the components relevant to the mechanical model were meshed for finite element analysis in this study. This includes the oval window membrane, round window membrane, cupulas, vestibular maculae, basilar membrane, reissner's membrane, utricle, saccule, semicircular canals, and cochlea. The mechanical model was meshed with a total of 414,629 tetrahedral elements and 90,696 nodes using the software Hypermesh. Mesh size convergence analysis was not conducted for this model due to the already very fine average element size of about 0.2 mm, more than sufficient in the case of previous models like those seen in (JUNFENG PAPER, GAN PAPER). However, applying mesh size convergence analysis to future iterations of the model may reduce the processing power and time necessary for simulation. Tissues of all kinds were modeled using the Solid185 element type while the fluids, endolymph and perilymph, were modeled using the Fluid40 element type. A fine ruled mesh composed of 8,738 elements was chosen for the basilar membrane. All components were initially meshed with quadrilateral elements in 2D then split to yield a model with surfaces represented entirely by triangular elements. Components were assigned their respective thicknesses as seen in TABLE XXX and given proper connectivity using the imprint, extend, and create element tools native to Hypermesh. 

-INCLUDE DIMENSIONS TABLE like in gans middle ear paper

\subsection{Material Properties}
There is no available published data on the material properties of chinchilla inner ear soft tissues. The material properties of chinchilla inner ear tissues in this model are based on data available for human ears. Ishhii et al., (1990) first measured the strength of the membranous labyrinth of the human inner ear using a microtension tester. The elastic properties of the membranous labyrinth tissues of humans, including RWM, RM, and BM at apical, middle, and basal turns, were subsequently reported by Ishii and Yamamoto (1995) to be 9.8 MPa, 34.2 MPa, 6.4 MPa, 6.0 MPa and 9.7 MPa, respectively. The only available mechanical properties of the RWM measured in dynamic ranges were reported by Zhang, et al. (2013) and Gan, et al. (2013). In our model, an elastic modulus of 0.35 MPa was used for the chinchilla middle ear  (Wang et al., 2016). The elastic modulus of the RM was reported as 60 MPa elsewhere (Bronzino, 2000). For the chinchilla RM in this model, 50 MPa was used as the Young’s modulus and 0.4 as the poison ratio. With respect to the BM, stiffness varies along its length from base to apex, as reported in literature (Naidu and Mountain, 2000); Emadi et al., 2004); Wittbrodt et al., 2006); and Teudt et al ,2007)). To simulate the tuning effect, the material properties of the BM was transiently varied from base to apex: the stiffness or Young’s modulus of the BM was gradually decreased from the base to the apex. The Young’s modulus at the base was assumed to be 50 MPa, linearly decreased to 15 MPa at the middle, and further decreased to 3 MPa at the apex of the BM. The damping coefficient $\beta$ for the BM was reported as a function of the BM length. In this study, the damping for the BM was assumed to increase with an exponential pattern from 2.5×10-6 at the base to 1×8-4 s at the apex. The rest of the membranous labyrinth in the SCC was modeled with a Young’s modulus of 3 kPa and Poisson’s ratio of 0.4 following what has been used previously in the SCC model developed by Iveren et.al (2017)

Mechanical properties of the cupula are scarcely reported in literature. Grant and Vanbuskirt first measured the stiffness of the cupula of White King pigeons (1976). The Young’s modulus of the human cupula was estimated at 5.4 Pa based on experiments using thick and thin bending membrane theory (Selva et al., 2009). In the human SCC model reported by Iversen and Rabbitt (2017), the shear stiffness of the cupula was assumed to be 1 Pa. In the simulation discussed in this paper, the cupula was assumed to be composed of isotropic and homogeneous material with a shear of 1 Pa and Poisson ratio of 0.4. 

Aside from the RWM, a density of 1000 kg/m3 was chosen for the soft tissues of the inner ear. Considering the higher elastic modulus of the chinchilla RWM, the RWM may have a higher proportion of fibers than those of other inner ear soft tissues. The density of the RWM was assumed to be 1500 kg/m3, which is the same as that of the human RWM reported by Bronzino et al. (2000). To our best knowledge, there is no published description of inner ear bone density for chinchillas. In the human cochlea model described in Gan’s paper (Gan et al., 2007), the density of all solid structures was assumed to be 1200 kg/m3. Wang et al also concluded that the density of chinchilla bones is lower than that of humans (Wang et al., 2016). To simplify this problem, 1200 kg/m3 was used as the density and 14.1 GPa as the Young’s modulus for the BM support bones following what been formerly used in the human cochlea model reported by Wang et al.

Edit this as the model is completed and material properties are applied
Include a large table of all material properties and their origins


\subsection{Boundary Conditions}
The outer boundaries of the mechanical model lie at the division between the perilymphatic fluid of the inner ear and the bony labyrinth. This was modeled as fixed due to the relatively small amount of information available on chinchilla temporal bone material properties. Each surface between lymphatic fluid and a movable structure, such as the BM, RM and RWM, was defined in the model as a fluid-structure interface (FSI) where the acoustic pressure was coupled into structural analysis. For harmonic acoustic simulation the oval window membrane was subjected to XXXX (reference Junfeng's model to figure out what to do for analysis here). The vestibular response of the model was tested with X angular acceleration with displacement tracked for each cupula and x linear acceleration with displacement tracked for the vestibular maculae.

-Get help from Junfeng on this part, not sure what else to include here


\section{Results}
\subsection{Geometry}
A working mechanical model was constructed from accurate geometries for the bony and membranous labyrinths of the chinchilla inner ear with response evaluable at the basilar membrane, cupulas, and vestibular maculae. This is augmented by .stl models of the vestibulocochlear nerve, auditory cortex, and temporal bone structure for use in later electrical analysis. Features of the mechanical model which differentiate it from others include the presence of the utricle, saccule, vestibular maculae, and geometrically accurate cochlea. A FEW FIGURES OF COMPLETED MECHANICAL MODEL, MAIN FEATURES. Figure \ref{wm} denotes the locations of the round and oval window membranes.


Figure \ref{bonenerve} shows the course the facial nerve takes through the temporal bone, identified using CT scans which confirmed the pathways found by segmenting MRI images with corresponding lumens in the temporal bone. Figure \ref{nvs} shows a more detailed view of the nervous system and its attachments to the membranous labyrinth.


\subsection{FE Analysis}
NEED \newline - Graph from FE analysis like seen in Junfengs paper
- Maybe a figure showing the allocation of material properties by color
- Will have graphs/images for each trial, acoustic harmonic and 2 accelerations
- Reference Junfeng and Gans papers to word and report important values
- Get Junfengs help on this part



\section{Discussion}\label{discussion}
This study features a finite element model constructed by coregistering $\mu$CT and $\mu$MRI scans of the chinchilla bulla. The chinchilla was chosen as the basis for this FE model as it is widely accepted as one of the closest analogs to the human inner ear due to its similar number of turns in the cochlea, structure of semicircular canals, and singular primary crista. \cite{trevino:chinchmodel} The chinchilla is perhaps a better candidate than other rodents, especially microtype varieties, as evidenced in figure \ref{response2} which shows the cochlear response of a variety of rodents and is easily compared to figure \ref{response} displaying human and chinchilla cochlear response. Only the gerbil and chinchilla exhibit the desired response and range from and in frequency. \cite{mason}

Because the saccule is linked to the cochlear duct, propagation of waves due to angular motion of the SCC’s can cause perturbations of the base of the cochlea resulting in sensible alterations in hearing, with intense noises capable of producing a similar effect in the vestibular system. For this reason, it is vital that the membranous labyrinth be modeled in its entirety, linking the saccule and the ductus cochlearis with the canalus reuniens and allowing for analysis of their effects on one another. The greatest effect of the vestibular system and cochlea on the other can be predictably seen at the base of the cochlea and the utricular and saccular vestibules as they are the closest in proximity to one another. INSERT SOME REFERENCES TO OUR DATA HERE.  Studies with rodents have confirmed this effect \cite{mizrachi}

DISCUSSION ON RESULTS FROM MACULAE, CUPULAS

Geometric accuracy of the vestibular system and cochlea will allow for studies into the effects of numerous diseases of the inner ear by varying geometry and applying further inverse finite element analysis to compare with data recieved from both human and chinchilla tests subjects. These studies can include the effects of otitis media, inflammation of inner ear tissues, and genetic abnormalities. Completed nerve, temporal bone, and auditory cortex geometries can be substituted into the model as needed later to allow for analysis of multisystemic pathologies which are the result of both mechanical and electrical malfunctions in the ear. The accuracy of geometry in nervous tissue will allow for studies on the effect of procedures or implants on nerve function. One example is that of the installation of a cochlear implant, where electrodes could be optimized through reverse analysis of the finite element model and allow for new technologies which minimize the effect on face or tongue feeling. Inclusion of the temporal bone will allow for studies on the effects of concussive blasts and provide an opportunity for analysis of the inner ear in its damaged state.

This model is also capable of being attached to a developed model of the chinchilla middle and outer ear. After the behaviour of this inner ear model is concretely known the middle ear transfer function in various scenarious can be recorded to aid clinical diagnosis of ear disorders.

Some features that are important for vestibulocochlear function have been omitted and their inclusion could improve results in the future. Imaging was unable to capture the proper thicknesses of many membranes on its own. This could be improved using histology, OCR scanning technology, and mechanical testing of excised tissues from recently diceased animal specimens to ensure the proper elasticity and water content of the tissue. Another valuable addition would be the involvement of a seperate finite element model representing a sensory hair cell. This would make transient analysis of multisystemic pathologies feasible by translating deflection from the mechanical model into electrical impulses which could be studied in the electrical model. \cite{choi} 

what would I add
- nerves for electrical simulation
- temporal bone thickness and material properties
- attach middle and outer ear model already completed
   
\section{Acknowledgment}
This work is supported by .... (Dr. Dai, OU, more?)

\bibliographystyle{unsrt}
\bibliography{references}

\end{document}

% More sources: 
%1. Harrison J.M., Irving R. (1966) Visual and nonvisual auditory systems in mammals. Anatomical evidence indicates two kinds of auditory pathways and suggests two kinds of hearing in mammals. Science, 154, 738–743. [PubMed] [Google Scholar]
% 2. Henderson D. (1969) Temporal summation of acoustic signals by the chinchilla. J. Acoust. Soc. Am., 46, 474–475. [PubMed] [Google Scholar]
% 3. Margolis R.H., Schachern P.L., Hunter L.L. et al. (1995) Multifrequency tympanometry in chinchillas. Audiology, 34, 232–247. [PubMed] [Google Scholar]
% 4. Miller J.D. (1970) Audibility curve of the chinchilla. J. Acoust. Soc. Am., 48, 513–523. [PubMed] [Google Scholar] 
% 5. Shimoyama, M., Smith, J. R., De Pons, J., Tutaj, M., Khampang, P., Hong, W., Erbe, C. B., Ehrlich, G. D., Bakaletz, L. O., & Kerschner, J. E. (2016). The Chinchilla Research Resource Database: resource for an otolaryngology disease model. Database : the journal of biological databases and curation, 2016, baw073. https://doi.org/10.1093/database/baw073
